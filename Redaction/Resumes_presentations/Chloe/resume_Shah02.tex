\documentclass[a4paper]{article}

\usepackage[francais]{babel}
\usepackage[T1]{fontenc}
\usepackage[applemac]{inputenc}

\usepackage{geometry}
\usepackage{graphicx}
\usepackage{listings}
\usepackage{amssymb}
\usepackage{setspace}
\usepackage{lscape}
\usepackage[final]{pdfpages}
\usepackage{algorithm, algorithmic}
\usepackage{amsmath}
\usepackage[colorlinks=true]{hyperref}
\hypersetup{urlcolor=blue,linkcolor=black,colorlinks=true} 


\geometry{a4paper,twoside,left=2.5cm,right=2.5cm,marginparwidth=1.2cm,marginparsep=3mm,top=2.5cm,bottom=2.5cm}


\normalsize
\setlength{\parskip}{8mm plus2mm minus2mm}


\begin{document}

Chlo� Desdouits \hfill 12 f�vrier 2012


{\centering \Large \bfseries R�sum� de l'article de Rahul C. Shah et Jan M. Rabaey\\
Energy Aware Routing for Low Energy Ad Hoc Sensor Networks \par}

Cet article traite des diff�rents aspects du projet PicoRadio. Ce projet consiste � utiliser un r�seau de capteurs (PicoNodes) pour g�rer la consommation �nerg�tique d'un b�timent. Ces PicoNodes sont des petits capteurs de moins d'un centim�tre cube qui co�tent moins d'un dollar.

Le d�but de cet article traite des couches physique, liaison et r�seau des PicoNodes. La couche r�seau s'occupe des adresses des capteurs. L'auteur a choisi de former ces adresses par le triplet suivant : <Position g�ographique, Type de n�ud (capteur, contr�leur , d�clencheur), Sous-type de n�ud (capteur de temp�rature, humidit��)>.

Toute la deuxi�me moiti� de l'article traite des protocoles de routage. Il existe deux types de protocoles : les proactifs et les r�actifs.
	Les protocoles proactifs font en sorte que tous les n�uds ait toujours une vision correcte de l'ensemble du r�seau. Ainsi, tous les changements de topologie sont imm�diatement transmis � tout le r�seau. Exemples de protocoles proactifs : DSDV (Destination-Sequenced Distance-Vector Routing protocol) \cite{Perkins1994} ; Link-state Routing \cite{Jacquet2001}.
	Les protocoles r�actifs, quant � eux, ne d�couvrent un chemin entre deux n�uds qu'en cas de besoin. Certains protocoles initient la d�couverte d'un chemin depuis la source vers la destination ; d'autre partent de la destination vers la source. Exemples : AODV (Ad-Hoc On-Demand Distance Vector Routing) \cite{Perkins1999} ; DSR (Dynamic Source Routing) \cite{Johnson2001} ; Directed Diffusion \cite{Intanagonwiwat2003}.
	
L'auteur propose un nouveau protocole de routage : Energy Aware Routing. Ce protocole est r�actif et les requ�tes sont initi�es depuis la destination. Il se d�cline en trois phases : la phase d'initialisation qui permet de trouver tous les chemins de la source � la destination ; la phase de communication de donn�es ; et la maintenance des routes. Les deux premi�res phases sont d�taill�es dans les algorithmes ci-dessous.

\begin{algorithm}[h]
\caption{Setup phase}
\label{algo_EAR_sp}
\begin{algorithmic}

\STATE $Cost(N_D) = 0$

\COMMENT{Flood the network from $N_D$ to $N_S$ :}
\FOR{each intermediate node $N_i$}
	\FOR{each neighbor $N_j$ of $N_i$}
		\IF{$d(N_i,N_S) \geq d(N_j, N_S)$ and $d(N_i,N_D) \leq d(N_j, N_D)$}
			\STATE Forward the request from $N_i$ to $N_j$
		\ENDIF
	\ENDFOR
\ENDFOR
\STATE
\FOR{each intermediate node $N_i$}
	\FOR{each neighbor $N_j$ of $N_i$}
		\STATE On receiving a request :
		\STATE $C_{N_j,N_i} = Cost(N_i)+Metric(N_j,N_i)$
		\STATE $FT_j = \{i | C_{N_j,N_i} \leq \alpha (\underset{k}{min}\ C_{N_j,N_k})\}$
		\STATE
		\STATE $P_{N_j,N_i} = \dfrac{\dfrac{1}{C_{N_j,N_i}}}{\sum \limits_{k \in FT_j} \dfrac{1}{C_{N_j,N_k}}}$
		\STATE
		\STATE $Cost(N_j) = \sum \limits_{i \in FT_j} {P_{N_j,N_i} C_{N_j,N_i}}$
	\ENDFOR
\ENDFOR
\end{algorithmic}
\end{algorithm}


\begin{algorithm}[h]
\caption{Data communication phase}
\label{algo_EAR_dcp}
\begin{algorithmic}

\STATE $N_S$ chooses randomly a number
\STATE This number points a neighbor in the forwarding table $FT_S$ out
\STATE Send the packet from $N_S$ to this neighbor $N_j$
\STATE Do the same thing for all intermediate nodes until the packet reaches the destination

\end{algorithmic}
\end{algorithm}


\bibliographystyle{plain}
\bibliography{../../Bib}

\end{document}
