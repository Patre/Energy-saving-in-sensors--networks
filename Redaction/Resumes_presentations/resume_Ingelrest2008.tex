\documentclass[a4paper]{article}

\usepackage[francais]{babel}
\usepackage[T1]{fontenc}
\usepackage[applemac]{inputenc}

\usepackage{geometry}
\usepackage{graphicx}
\usepackage{listings}
\usepackage{amssymb}
\usepackage{setspace}
\usepackage{lscape}
\usepackage[final]{pdfpages}
\usepackage{algorithm, algorithmic}
\usepackage{amsmath}
\usepackage[colorlinks=true]{hyperref}
\hypersetup{urlcolor=blue,linkcolor=black,colorlinks=true} 


\geometry{a4paper,twoside,left=2.5cm,right=2.5cm,marginparwidth=1.2cm,marginparsep=3mm,top=2.5cm,bottom=2.5cm}


\normalsize
\setlength{\parskip}{8mm plus2mm minus2mm}


\begin{document}

Chlo� Desdouits \hfill 12 f�vrier 2012


{\centering \Large \bfseries R�sum� de l'article de Fran�ois Ingelrest and David Simplot-Ryl\\
Localized Broadcast Incremental Power Protocol for Wireless Ad Hoc Networks \par}


\begin{algorithm}[h]
\caption{LBIP setup phase}
\label{algo_LBIP_sp}
\begin{algorithmic}

\FOR{each node $N_i$}
	\STATE Broadcast a HELLO message
\ENDFOR

\STATE On receiving a message, put the sender in the 1-hop neighborhood

\FOR{each node $N_i$}
	\STATE Broadcast a HELLO message which contains the $N_i$ 1-hop neighborhood to compute the $N_i$ 2-hops neighborhood
\ENDFOR

\COMMENT{each node has a BIP tree of its 2-hops neighborhood}

\end{algorithmic}
\end{algorithm}

a

\begin{algorithm}[h]
\caption{LBIP}
\label{algo_LBIP}
\begin{algorithmic}

\STATE The source node $s$ sends a packet which contains instructions for relayer nodes
\IF{A node $u$ receives a packet from a node $v$ for the first time :}
	\IF{The packet contains some instructions for u}
		\STATE $u$ starts constructing a BIP tree within its own 2-hop neighborhood with ranges previously computed
	\ENDIF
\ENDIF

\end{algorithmic}
\end{algorithm}



%\bibliographystyle{plain}
%\bibliography{../../Bib}

\end{document}
