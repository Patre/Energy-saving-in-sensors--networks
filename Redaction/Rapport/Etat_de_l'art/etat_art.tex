% !TEX encoding = UTF-8 Unicode
% !TEX root = ../rapport.tex

\chapter{État de l'art}\label{etat_art}

\section{Modélisation des réseaux de capteurs sans fil}

\subsection{Critères pratiques}\label{modelePratique}

Différents modèles sont possibles:

% TODO : citer les articles qui utilisent telle ou telle modélisation
La modélisation de base des réseaux de capteurs sans fil \textbf{$M_1$}:

\begin{itemize}
 \item \textit{Uniforme.} Tout les capteurs sont identiques (batterie, portée, capacité de calcul..).
 \item \textit{Connexe.} Le réseau est initialement connexe (chaque capteur est lié directement ou indirectement à tout les autres).
 \item \textit{Plan} Dans un plan euclidien à deux dimensions (distance euclidienne).
 \item \textit{Statique.} Sans mobilité des capteurs : nous supposerons que les capteurs sont immobiles.
 \item \textit{Sans ajout de capteurs.} Le réseau comprend un nombre fixe de capteurs $n$. Aucun ajout de capteurs en cours de fonctionnement n'est possible.
 \item \textit{Transmission.} Dans des conditions de transmission de message idéale: aucune interférence entre les messages, pas de perturbations des ondes, système d'identifiants unique.
 \item \textit{Fiabilité.} Les capteurs sont fiables, aucunes pannes ne sont possibles.
 \item \textit{Energie.} Chaque capteur a une énergie initiale $\beta$ donnée.Un modèle décrit la consommation énergetique. Un capteur est éliminé lorsqu'il n'a plus d'énergie ou que son énergie 
 restante ne permet plus aucun envoi de message. 
 \item \textit{Egalité.} Chaque site peut a tout moment débuter une procédure de broadcast. Une loi de probabilité modélise 
 ce phénomène(loi de poisson).
 \item \textit{Position.} Chaque site connait sa position absolue.  \\
\end{itemize}

D'autres modèles plus complexes mais plus réalistes prennent en compte:
  
\begin{itemize}
   
 \item L'ajout de capteurs: le réseau comprend un nombre variable de capteurs $n$. Les ajouts de capteurs en cours de fonctionnement sont possibles.
 \item Les capteurs peuvent tomber en panne en raisons de divers facteurs. Une loi de probabilité modélise ce phénomène.   
 \item La mobilité des capteurs. La position de chaque capteur varie au cours du temps.
 \item Les capteur ne sont pas forcement identiques (batterie, portée, capacité de calcul, connaissance du reseau...).
 \item Espace en 3 dimensions.
 \item Interferences radio.
 \item Modele de consommation energetique complexe.
 \item ...
\end{itemize}


\subsection{Modèle d'un réseau de capteurs : un graphe}

 \paragraph*{} Un WSN peut être représenté par un graphe $G= (V,E,\gamma)$ ou $V$ est un ensemble de noeuds (capteurs), $\gamma$ le rayon d'émission maximum et $E \subseteq V \textsuperscript{2}$ l'ensemble des arêtes représentant les communications possibles entre les capteurs: $(u,v)$ appartient à $E$ signifie que $u$ peut envoyer un  message à $v$. On note $ n=|V| $ la taille du WSN. En fait les éléments de E dépendent de la
 position des capteurs ainsi que de leur portée. Nous supposerons que tout les capteurs ont la même portée maximale notée $\gamma$. 

%%%%%%%%%%%%%%%%%%%%%%%%%%%%%%%%%%  Distance

\begin{mydef}
Nous noterons $ij$ l'arête allant de $i$ à $j$. \\
Nous noterons $d_e(u,v)$ la \textit{distance euclidienne} dans $\mathbb{R} \textsuperscript{2}$ entre $u$ et $v$:
$$E = \{ (u,v) \in V ^{2} \mid d(u,v) \leq \gamma \}$$
Nous noterons $d_G(u,v)$ la \textit{distance} entre $ u $ et $ v $ : $d_G(u,v)= \min\limits_{k \in \mathbb{N}}(k \mid v \in N_k(u))$
\end{mydef}


%%%%%%%%%%%%%%%%%%%%%%%%%%%%%%%%%%   Graphe unité
\begin{mydef}
 On appelera $G= (V,E,\gamma)$ le \textit{graphe unité} du WSN et $\gamma$ son rayon de communication.
\end{mydef}



%%%%%%%%%%%%%%%%%%%%%%%%%%%%%%%%%%   Voisinage
\begin{mydef}
Nous noterons le 1-voisinage de $u$ : $N_1(u) = \{ v \in V  \mid (u,v) \in E \}$ \\
Nous noterons le 2-voisinage de $u$ : $N_2(u) = \{ v \in V \mid  \exists w \in V :\{(u,w);(w,v)\} \in E ^2\}$ \\
Nous noterons le $k$-voisinage de $u$, $k \in \mathbb{N} : N_k(u) = \{ v \in V  \mid \exists $ un chemin $c (u,v): |c| \leq k\}$  Nous parlerons de 1- 2- et k-voisins de $i$ pour désigner des noeuds appartenant respectivement
 à $N_1(i), N_2(i),N_k(i)$. \\
Soit $A \subseteq V$, on note $N(A) = \{ v \in V\textbackslash  A \mid \forall u\in A,(u,v) \in E \}$ \\
Le degré de $ u $ est le nombre  $N(u)=|N_1(u)|$.\\
\end{mydef}

%%%%%%%%%%%%%%%%%%%%%%%%%%%%%%%%%%   Diametre
\begin{mydef}
Nous noterons $diametre_G= \max\limits_{i,j\in \textlbrackdbl 1,n \textrbrackdbl,i<j} (d_G(i,j))$.
\end{mydef}
 

\begin{figure}[H]
\centering
\includegraphics[scale=0.5]{Etat_de_l'art/source/graph1.png}
\caption{Graphe unité}
\end{figure} 



%%%%%%%%%%%%%%%%%%%%%%%%%%%%%%%%%  Densité, distance moyenne
\begin{mydef}
 
 Le degré de $G$ est la moyenne des degrés:$$N_G=\sum_{i=1}^n{\frac1n N(i)}$$\\
 La densité de $G$ est le nombre $D_G=N_G/diametre_G$\\
 La distance de $G$ est la moyenne des distances entre toutes paires de sommets:$$d_G=\sum_{i,j\in \textlbrackdbl 1,n \textrbrackdbl,i<j}{\frac1n d_G(i,j)}$$
 La distance euclidienne de $G$ est la moyenne des distances euclidienne entre toutes paires de sommets:$$d_e(G)=\sum_{i,j\in \textlbrackdbl 1,n \textrbrackdbl,i<j}{\frac1n d_e1(i,j)}$$

\end{mydef}


\subsection{Modèle énergétique}
\subsubsection{Energie d'un capteur}
Dans \cite{Dong2005}, Dong présente les deux modèles de consommation énergétique communément utilisés.
\paragraph{The Packet based model.}
Nous utiliserons pour notre analyse le modèle de consommation d'énergie idéal suivant:
Nous considèrerons que chaque capteur $i$ a une énergie initiale $E_{init}=\beta$.
L'envoi de message est le seul facteur de perte d'énergie. L'énergie consommée lors de la réception de message, l'acquisition et traitement des informations sera considérée comme négligeable.
Tous les capteurs $i$ offrant les mêmes caractéristiques, ils peuvent modifier leur rayon d'émission $r_i$ entre $r_i=0$ et $r_i=\gamma$.
Nous noterons $E_i$ l'énergie restante de $i$.
L'envoie d'un message de $i$ avec un rayon $r$ coute $$ E(r)= \begin{cases} r^\alpha + c & \text{si }i\neq j \\ 0 & \text{sinon}  \end{cases}$$
L'envoi d'un message de $i$ à $j$ ($d_e(i,j)\leq \gamma$) coûte  $ E_{ij}=E(d_e(i,j))$.

\paragraph{The Time based model}
Cependant,un autre modèle plus réaliste prend en compte l'énergie de réception des message, de traitement ainsi que d'écoute inactive du réseau (mode veille).
En effet, dans \cite{Kasten2001}, Kasten souligne le fait que souvent, la réception, l'écoute et le traitement consomme en moyenne autant d'énergie que la transmission de message.
Dans de nombreuse topologie, si la fréquence des broadcasts est faible, beaucoup de capteurs vont être a cours d'énergie avant même d'avoir pu transmettre des messages.
Dans beaucoup d'application, la densité du réseau étant élevée, le cout propre aux transmissions de message est relativement faible tandis que le cout de réception des message est élevé puisque chaque capteur traite les message de son voisinage
qui en l'occurance est grand.



\paragraph{Energie globale}
\begin{mydef}
 L'énergie potentielle de G est la somme des énergies des capteurs :$$E_G=\sum_{i=1}^n{E_i}$$
 La consommation de  G est :$$C_G=n\beta - E_G$$
 Le coût moyen de transmition de  G est :$$c_G=E(d_e(G))$$
 Le cout moyen d'un broadcast est $C(1)$.\\
 Le cout moyen de $k$ broadcast est $C(k)$


\end{mydef}



\subsection{Durée de vie du réseau}
\subsubsection{Problématique}


Dans un réseau de capteurs sans fil, la contrainte majeur est l'efficacité de l'algorithme utilisé en matière de consommation énergétique. En effet la principale caractéristique des capteurs
est leur petite taille et leur micro-batterie. Dans la majeur partie des cas, remplacer les batteries est impossible. Cependant la durée de vie d'un WSN est difficile à définir et a mesurer.
Il n'y a pas de définition absolue. Elle indique combien de temps le réseau sera 'efficace' par rapport a l'application donnée (nombre de broadcasts effectués, connexité, pourcentage de noeuds vivants...).

Dans la littérature, deux approches apparaissent clairement en matière de maximisation de la durée de vie des WSNs. Une approche indirecte consiste à minimiser la consommation d'énergie de façon locale tandis que l'autre a pour but 
de maximiser la directement la durée de vie du réseau de façon plus globale. Bien que l'approche indirecte puisse améliorer la durée de vie du réseau, elle ne suffit pas à elle seule à appréhender le problème de la durée de vie.
Parmi ces approches figurent  par exemple le fait de maximiser le nombre de transmissions effectuées avant qu'un capteur ne meurt.

Dans \cite{Liang2002}, Liang prouve que le  problème THE MINIMUM-ENERGY BROADCAST
TREE PROBLEM est NP-complet par réduction à (3-CNF SAT):
\begin{myth}
Soient un WSN $G$ dans lequel chaque noeud a $k$ rayons de transmission possibles, une source $s$ et un entier positif $w$.
Déterminer s'il existe un arbre de broadcast de source $s$ tel que la somme des couts de transmission aux noeuds relais (qui ne sont pas des feuilles) soit inférieure à $w$ est NP-Complet.
\end{myth}
Ainsi, pour un broadcast donné il n'existe pas d'algorithme centralisé polynomial pour trouver l'arbre de diffusion optimale. Il est évident qu'il n'en existe pas de distribués. 
Dans \cite{Dong2005}, Dong prouve que le  problème BROADCAST LIFETIME est NP-Complet par réduction de (3DM):
\begin{myth}
Soient un WSN $G$, une source $s$ et un entier positif $k$.
Déterminer si G a assez d'énergie pour broadcaster $k$ messages a partir de $s$ est NP-Complet.
\end{myth}

De façon formelle, il est donc souvent impossible de calculer le nombre de broadcasts total réussis étant donné un réseau et un protocole (sauf quand le protocole est très simple $cf$ \ref{algosSansBalisage}). Nous analyserons 
donc \textit{les performances moyennes} des algorithmes quand cela est possible. Sinon, les simulations nous permettrons de mesurer les performances en fonction de différentes topologies.

\subsubsection{Différentes définitions}
Les définitions et critères de durée de vie d'un WSN sont tirés des articles \cite{Dietrich2009},\cite{Champ2009lifetime},\cite{Elleithy2011}. 

\begin{mylt}
Nombre moyen de transmissions réussies avant qu'un capteur n'ai plus de batterie (TTFF).
\end{mylt}
\begin{mylt}
Nombre moyen de transmissions réussies avant que le réseau ne perde sa connectivité.
\end{mylt}
\begin{mylt}
Nombre moyen de transmissions réussies jusqu'à ce qu'il ne reste $X$ pourcent de noeuds vivants.
\end{mylt}



{%
\newcommand{\mc}[3]{\multicolumn{#1}{#2}{#3}}
\begin{center}
\begin{tabular}{|c|l}\cline{1-1}
\mc{2}{c}{\textbf{Notations}}\\\hline
$G(V,E,\gamma)$ & \mc{1}{c|}{Graphe unité}\\\hline
$n$ & \mc{1}{c|}{Nombre de capteur}\\\hline
$\gamma$ & \mc{1}{c|}{Rayon d'émission maximum}\\\hline
$\alpha$ & \mc{1}{c|}{Constante de consommation énergetique}\\\hline
$c$ & \mc{1}{c|}{Constante de consommation énergetique}\\\hline
$\beta$ & \mc{1}{c|}{Energie initiale des capteur}\\\hline
$E_i$ & \mc{1}{c|}{Energie restante de $i$}\\\hline
$E_{ij}$ & \mc{1}{c|}{Cout d'envoie d'un message de $i$ à $j$}\\\hline
$N(u) $& \mc{1}{c|}{Degre de u}\\\hline
$d_G(i,j)$ & \mc{1}{c|}{Distance dans G entre $i$ et $j$}\\\hline
$d_e(i,j)$ & \mc{1}{c|}{Distance euclidienne entre $i$ et $j$}\\\hline
$N_k(u)$ & \mc{1}{c|}{Nombre de k-voisins de u }\\\hline
$N_G$ & \mc{1}{c|}{Degre de G}\\\hline
$d_G$ & \mc{1}{c|}{Distance de G}\\\hline
$d_e(G)$ & \mc{1}{c|}{Distance euclidienne de G}\\\hline
$diametre_G$ & \mc{1}{c|}{Diametre de G}\\\hline
$D_G$ & \mc{1}{c|}{Densité de G}\\\hline
$C(k)$ & \mc{1}{c|}{Cout de $k$ broadcasts}\\\hline

\end{tabular}
\end{center}
}%



\section{Algorithmes existants}\label{class}

Les réseaux de capteurs sans fils constituent un domaine de recherche récent et actif. Une grande quantité d'articles ont été publiés cette dernière décénnie. Nous n'en aborderons que quelques uns.


\subsection{Elements de classification}
Les éléments de classification cités ci-dessous sont inspirés des articles \cite{stojmenovic2004},\cite{ingelrest2005},\cite{wu2003}.


\subsubsection{Transmissions en broadcast vs transmissions en single-cast}
% TODO vs single-cast
Dans un WSN $G(V,E,\gamma)$, deux capteurs peuvent communiquer directement uniquement s'ils sont 1-voisins dans G. A cause de la perte de propagation des messages, le rayon de transmission est relativement limité, c'est pourquoi les communications doivent se faire par multi-sauts, parfois même si le destinataire est à distance 1 dans le graphe unité (pour des raisons énergétiques).

Pour établir une connexion entre deux noeuds non voisins, les messages doivent effectuer des sauts via des noeuds intermédiaires. Dans un large WSN, il est bien trop difficile pour un capteur voulant
transmettre un message à un autre de trouver une route, à cause de l'absence d'infrastructures. 
La procédure de broadcast est un mécanisme fondamental pour la propagation des données ainsi que pour la découverte de route. C'est pourquoi, concevoir des algorithmes efficaces et économes en énergie est un problème primordial dans les WSN. 

Algorithme à rayon d'émission fixe vs variable


\subsubsection{Phase d'initialisation des capteurs (balisage) vs beaconless (sans balisage)}
Un capteur 'prend naissance', c'est à dire débute son activité dès qu'il est en place dans le réseau et ce automatiquement. Deux choix sont possibles: soit le capteur est immédiatement opérationnel, soit il nécessite une phase d'initialisation. Il existe donc deux type d'algorithmes: sans balisage (beaconless) et avec balisage. Les algorithmes 'beaconless' ne nécessitent pas de phase d'initialisation ni de mise à jour. Les nouveaux capteurs sont immédiatement opérationnels mais n'ont aucune connaissance de leur environnement et notamment de leur voisinage dans G. 

Dans les algorithmes avec balisage, tout capteur, lorsqu'il 'prend naissance' commence par une procédure d'initialisation et stocke en mémoire un certain nombre d'informations (voisins, groupe, topologie locale...). Au cœur de l'algorithme, chaque noeud met périodiquement à jour ces informations. Pour ce faire chaque site envoie régulièrement à ses voisins un message de type
'Hello' contenant par exemple son Id, sa position, sa dominante connexe, son degré, ses voisins, etc.

\subsubsection{Algorithme global vs local (distribué)}
Quel type d'informations est utilisée dans l'algorithme: informations globale du reseau ou informations locales? La distinction entre global est local n'est pas toujours évidente. Des algorithmes centralisés peuvent etre implémentés
 d'une maniere distribuée en décidant par exemple d'un noeud possedant la connaissance globale du réseau.Sinon, au travers d'un échange séquentiel d'informations tres localisées (1 ou 2-voisinage par exemple), chaque noeud peut combiner sa 
connaissance avec celle de ses voisins et ainsi obtenir une vision globale du réseau. Cepandant, une telle phase de propagation coute tres cher en terme de nombre de messages échangés et de temps.
Si le reseau est dynamique (mobilité des capteurs), maintenir une connaissance locale du reseau devient plus complexe tandis que tenir a jour la topologie globale de celui-ci devient impossible par le moyen cité précedement.Ainsi, 
la quantité et la nature des informations nécessaire au déroulement de l'algorithme sont une bonne mesure de la capacité d'adaptation du protocol à un environement dynamique.   

\begin{enumerate}
 \item \textbf{Global}:         protocol de broadcast, centralisé ou distribué nécessitant une connaissance globale du reseau(ex: BIP).
 \item \textbf{Quasi-global}:   protocol distribué de broadcast nécessitant une connaissance quasi-globale du reseau.
 \item \textbf{Quasi-local}:    protocol distribué nécessitant une connaissance du réseau principalement locale et occasionellement globale.
(ex: Cluster networks: tandis que les groupes peuvent etres construits de maniere locale , des reactions en chaines peuvent arrivées).
 \item \textbf{Local}: 		protocol distribué nécessitant une connaissance tres locale du réseau.Tout les algorithmes de 1 ou 2-voisinage appartienent à cette catégorie.
\end{enumerate}

Pour qu'un protocole soit extensible, il doit être avant tout distribué et local: le comportement de chaque noeud bien qu'il nécessite qu'une connaissance locale du réseau, permet d'atteindre l'objectif global. Il est facile d'admettre que l'extensibilité d'un WSN est inversement proportionnelle à la localité du protocole.


\subsubsection{Algorithme déterministe vs probabiliste}
Un protocole de broadcast peut utiliser des fonctions probabilistes pour prendre certaines décisions de manière aléatoire.





\subsection{Algorithmes sans balisage}\label{algosSansBalisage}

\paragraph{Bling Flooding}

Le blind flooding ou broadcast aveugle est un algortihme glouton de broadcast. Lors de la reception d'un message par un noeud, si c'est la premiere fois qu'il le recoit, il le broadcast à ses voivins(avec le rayon maximum), sinon il
 ne fait rien.

\textbf{Consomation et durée de vie}: Le cout global d'un broadcast est exactement le cout moyen puisque chaque noeud $i$ envoie un unique message <$M$,$\gamma$> par broadcast.
Le cout global d'un broadcast est $$C(1) = n \cdot E( \gamma )= n\cdot \gamma^\alpha +  n\cdot c $$
Le cout global de $k$ broadcast est $$C(k) = k\cdot n \cdot E( \gamma )= k\cdot n \cdot \gamma^\alpha +  k \cdot n\cdot c $$
La durée de vie du reseau est exactement de $$LT(Blind Flooding)=\lfloor \frac{\beta}{n\cdot \gamma^\alpha +  n\cdot c} \rceil$$
\paragraph{Probabilistic Flooding}

Afin d'eviter le redondance et les collisions, une idée est que chaque noeud retransmet le message suivant une probabilité $P$ lorsqu'il le reçoit pour la premiere fois.
Si $P=1$ cela equivaut au blind flooding.

\textbf{Consomation et durée de vie}: Chaque noeud ayant un probabilité $P$ de retransmettre le message <$M$,$\gamma$>, en moyenne $P\cdot n$ envoie ce message lors d'un broadcast.
Le cout moyen d'un broadcast est $$C(1) = P\cdot n \cdot E( \gamma ) $$
Le cout global de $k$ broadcast est $$C(k) = P\cdot k\cdot n \cdot E( \gamma ) $$
La durée de vie moyenne du reseau est  $$TTFF(Probabilistic Flooding)=\lfloor \frac{\beta}{P \cdot (n\cdot \gamma^\alpha +  n\cdot c)} \rceil$$



\paragraph{Algorithme ABBA}
'Area-based beaconless reliable broadcasting in
sensor networks'\cite{Abba2006}.

L'idée maitresse de ABBA est assez simple. Nous supposons que les noeuds n'ont aucunes connaissance de leur voisinage. Cepandant ceux ci connaissent leurs position géographique (GPS par exemple).
A la premiere réception d'un message, le noeud initialise un timer. Avant expiration du timer, le noeud peut recevoir d'autre copie du meme message. Chaque capteur a un rayon d'emission fixe $R$
et couvre une zone circulaire.
%\begin{figure}[H]
%\centering
%\includegraphics{Etat_de_l'art/source/abba.png}
%\caption{ }
%\end{figure} 
 Si un noeud $u$ recoit le meme message de differente source et que ces sources couvrent sa zone, alors $u$ ne retransmet pas le message.
Cela signifie que chaqun de ses voisins potentiel a déjà reçu le message. Si un noeud est couvert avant que son timer n'expire, il ne fait rien. Sinon il retransmet.
Comme chaque couverture est un cercle de rayon $R$, le critere de couverture peut etre simplifié: au lieu de s'interesser au disque entier, un noeud verifie uniquement si le perimetre $\pi$ de sa zone est couvert par 
ses voisins lui ayant envoyé le message. 

\begin{algorithm}[h]
\caption{ABBA}
\label{ABBA}
\begin{algorithmic}
\REQUIRE:
Un noeud source $s$, un message $M$
\STATE $s$ envoie  <$M$,$s$,$position_s$,$R$> ;
\STATE Reception  par  $u$ de  <$M$,$v$,$position_v$,$R$>
\STATE $z_u\leftarrow ensemble_{vide}$ ;
\STATE Calculer la zone $z$ couverte par $v$ lors de la transmittion;
\STATE Déclancher le timer;
\REPEAT
    \STATE Attendre la reception d'un autre copie de $M$ ou que le timer expire;
    \IF {Une copie de $M$ est reçue}
	\STATE Mettre à jour $z_u$;
	\STATE reinitialiser le timer;
    \ENDIF
\UNTIL{timer expire;}
\IF {$\pi \nsubseteq z_u$}
    \STATE Retransmetre;
\ENDIF
     \STATE Ignorer les autres copies de M;

\end{algorithmic}
\end{algorithm}


\subsection{Algorthimes globaux}
\paragraph{Algorithme BIP}
Article \cite{Wieselthier2000}

  Rappelons les desavantages des transmitions longue portée: interférences, cout energetique ( le modele de consommation energetique des noeuds est non linéaire par rapport au rayon à cause de l'attenuation du signal radio).
C'est pourquoi il est necessaire de trouver le bon compromis entre le rayon de transmition et le nombre de messages circulant: un large rayon de transmittion coute cher mais atteind beaucoup de neouds; un court rayon cout tres peu cher mais 
augmente le nombre de messages. 

'Boadcast Incremental Protocol' et glouton et centralisé.Il se base sur l'algorithme de Prim: un algorithme permettant de construire un arbre couvrant minimal d'un graphe. Le principe de l'algorithme de Prim est de construire l'arbre couvrant minimal arête par arête: pour ajouter une 
arête à un arbre partiellement construit, il considère l'ensemble des arêtes dont une extrémité est
connectée à l'arbre déjà construit, et l'autre extrémité ne l'est pas, et
il choisit dans cet ensemble une arête de poids minimal qu'il ajoute à l'arbre. L'algorithme commence avec un arbre couvrant contenant un noeud et zéro arêtes, et ajoute successivement $n-1$ arêtes.

La formation d'un arbre couvrant de poids minimun dans BIP suit le meme principe dans le sens ou les aretes sont une par une ajoutées à l'arbre.
En fait, Bip utilise l'algorithme de Prim avec une difference fondamentale: au lieu d'utiliser des couts fixes $P_{ij}$ sur les aretes (demeurant inchangés au court de la procédure),
Bip actualise dynamiquement ces couts $P'_{ij}$ à chaques étapes ($i.e$ à chaque ajout d'une arete), ce qui traduit le fait que le cout d'ajout d'une arete depends des noeuds déja dans l'abre:

$$ \forall i \in BIP, \forall j \notin BIP, P'_{ij}=P_{ij}-P(i)$$
ou $P_{ij}$ est le cout reel de transmition et $P(i)$ le cout de broadcast de $i$ dans l'arbre ( $P(i)=0$ si $i$ est une feuille, $P(i)=\max\limits_{j\in N_1(i)\bigcap BIP}(E_ij)$ sinon ). $P'_{ij}$ represente donc le cout d'ajout de $j$ par un noeud $i$ appartenant au sous ensemble des feuilles
directes de $i$(ou $i$ lui meme si c'est une feuille). La paire $\{i,j\}$ minimisant $P'_{ij}$ est selectionnée et $i$ transmet à $j$. Ainsi, une nouveau arete est ajoutée a chaque etape de l'algorithme.\\



\begin{algorithm}[h]
\caption{Procédure de construction du BIP-Tree}
\label{algo_BIP_tree}
\begin{algorithmic}
\STATE ENTREES  $G=(V,E)$ un graphe connexe, $s$ une source, un cout de transmition $P_{ij}$
\STATE SORTIE  Arbre BIP de racine s
\STATE B : ENSEMBLE des arêtes de l'arbre
\STATE  $B \leftarrow arbre_{vide}$
\STATE Marquer $s$
\STATE Creer les nouveaux poids: $\forall j \notin BIP, P'_{sj}=P_{sj}-P(s)=P_{sj}$
\WHILE {il existe un sommet non marqué adjacent à un sommet marqué }
   \STATE Mettre à jour les poids:  $ \forall i \in BIP, \forall j \notin BIP, P'_{ij}=P'_{ij}-P(i)$
   \STATE Sélectionner un sommet j non marqué adjacent à un sommet marqué i tel que (i,j) est l'arête sortante de plus faible poids $P'_{ij}$
   \STATE $B := B\bigcup   {(i,j)}$
   \STATE Marquer $j$
\ENDWHILE
\STATE Retourner $B=(V,B)$
\end{algorithmic}
\end{algorithm}

Contairement a Prim qui garantie l'optimalité de l'abre couvrant en termes de cout total,
Bip ne construit pas forcement un arbre de poids minimum.Cepandant, contrairemnt à Prim, Bip exploite l'avantage notoire du mulicast dût aux transmitions radio. Un fois cet abre construit, le broadcast se fait naturellement via celui-ci.

%\begin{figure}[H]
%\centering
%\includegraphics[scale=0.3]{Etat_de_l'art/source/Dessin2.pdf}
%\caption{ Arbre Prim(gauche); Arbre Bip(droite)}
%\end{figure} 

\begin{algorithm}[h]
\caption{BIP}
\label{algo_BIP}
\begin{algorithmic}
\STATE ENTREES  $G=(V,E)$ un graphe connexe, $s$ une source, un message $M$
\STATE SORTIE  BIP Broadcast
\STATE $s$ envoie $M$ à ses fils dans son Bip-tree
\STATE Lors de reception de $M$ par $i$:
  \IF {$i$ est un noeud} 
    \STATE Retransmettre $M$ à ses fils sinon ne rien faire
\ENDIF
\end{algorithmic}
\end{algorithm}




\subsection{Algorithmes avec balisage}
Comme expliqué en 2.1.3, dans les algorithmes avec balisage, tout capteur, lorsqu'il 'prend naissance' commence par une procédure d'initialisation et stocke en 
mémoire un certain nombre d'informations(voisins,groupe,topologie locale...). Au cour de l'algorithme, chaque noeud met périodiquement à jour ces informations. Pour ce faire chaque site envoie regulierement a ses voisins un message de type
'Hello' contenant par exemple son Id, sa position, sa dominante connexe, son degres, ses voisins, etc.\\
\paragraph{Decouverte 1,2-voisinnage}

La connaissance du 2-voisinage est un tres bon compromis conservant la localité du protocoles tout en minimisant le nombre de messages.


\begin{algorithm}[h]
\caption{Decouverte k-voisinnage}
\label{algo_k_voisinnage}
\begin{algorithmic}

\FOR{chaque noeud $i$}
	\STATE Broadcaster un message de type <HELLO,$i$> avec un rayon de transmittion $\gamma$
\ENDFOR

\STATE A la réception de <HELLO,$j$> , ajouter $j$ a son 1-voisinage.
\STATE Attendre $\Delta$
\FOR{chaque noeud $i$}
	\STATE Broadcaster <HELLO,$N_1(i)$,$i$>, message contenant $N_1(i)$ le 1-voisinage de $i$
	\STATE A la réception de <HELLO,$N_1(j)$,$j$> , ajouter $N_1(j)$ a son 2-voisinage.
	
\ENDFOR
\end{algorithmic}
\end{algorithm}

Complexité en messages: $O(2n)$.


\paragraph{Algorithme LBIP}  `Localized Broadcast Incremental Power Protocol for Wireless Ad Hoc Networks': article \cite{Ingelrest2004}
LBIP est l'application local de BIP: au lieu de construire l'arbre de diffusion BIP de facon centralisé,il le construit de facon locale.
Chaque site lorsqu'il recoit un message à retransmettre calcul son BIP-tree sur son 2-voisinage et diffuse le message par l'intermediaire de celui ci.

\begin{algorithm}[h]
\caption{LBIP}
\label{algo_LBIP}
\begin{algorithmic}
\STATE ENTREES  $G=(V,E)$ un graphe connexe, $s$ une source, un message $M$
\STATE SORTIE  LBIP Broadcast
\STATE REQUIE  Connaissance du 2-voisinage
\STATE $s$ calcul son arbre BIP($N_2(s),s,E_{ij}$) et diffuse <M,$s$> à ses fils
\IF{ $u$ recoit <M,$v$> :}
	\IF{Le paquet contient des instruction pour $u$}
		\STATE $u$ construit BIP($N_2(u),u,M$) est retransmet le message à ses fils
	\ENDIF
\ENDIF
\end{algorithmic}
\end{algorithm}

%\begin{figure}[H]
%\centering
%\includegraphics[scale=0.65]{Etat_de_l'art/source/LBIP.png}
%\caption{ }
%\end{figure} 


\paragraph{Algorithme DLBIP}
`Dynamic Localized Broadcast Incremental Power Protocol for Wireless Ad Hoc Networks': article\cite{Champ2009DLBIP}.
DLBIP est une amélioration de LBIP. Le principe est de repartir l'energie consommée lors d'un broadcast en empreintant des chemins (arbre couvrants) différents en fonction de l'energie restantes des noeuds qui vont servir de relais.
A chaque broadcast, l'abre de diffusion est recalculé sur le meme graphe mais avec de nouveaux poids dépendants de l'energie de communication entre les noeuds ainsi que de l'energie propre à chaque noeud. Ce protocols permet que
le niveau d'energie propre à chaque noeud baisse globalement de façon identique et ainsi retarder un maximum la panne d'un capteur par manque d'energie.
DLBIP actualise dynamiquement ces couts $P'_{ij}$ à chaques broadcast:
$$ \forall i,j \in V, P'_{ij}=\frac{E_{ij}}{E_i}$$
ou $E_{ij}$ est le cout reel de transmition et $E_i$ l'energie restante de $i$.


\begin{algorithm}[h]
\caption{DLBIP}
\label{algo_DLBIP}
\begin{algorithmic}
\STATE ENTREES  $G=(V,E)$ un graphe connexe, $s$ une source, un message $M$
\STATE SORTIE  DLBIP Broadcast
\STATE REQUIE  Connaissance du 2-voisinage
\STATE $s$ calcul son BIP($N_2(s),s,P'_{ij}$) et diffuse <M,$s$> à ses fils
\IF{ $u$ recoit <M,$v$> :}
	\IF{Le paquet contient des instruction pour $u$}
		\STATE $u$ construit BIP($N_2(u),u,P'_{ij}$) est retransmet le message à ses fils
	\ENDIF
\ENDIF
\end{algorithmic}
\end{algorithm}



\paragraph{Algorithme RRS}
\cite{Cartigny2003RNG}


\subsubsection{Algorithmes hexagonaux}


\paragraph{Rayon Optimal}


\begin{myth}
Sans chevauchement et sans vide, un plan ne peut etre découpé de facon uniforme que par des polygones reguliers de type triangle, carrés ou hexagones.
\end{myth}
\begin{proof}
 Soit $m$ le nombre de sommet d'un $m$-polygone et $n$ le nombre de $m$-polygones nécessaire pour couvrir $2\pi$ degrés
Nous avons: $$\frac{(m-2)n\pi}{m}=2\pi \Leftrightarrow (m-2)(n-2)=4$$
Comme $n$ et $m$ sont des entiers, les seuls solutions sont: $$(m-2;n-2)\in\{(1,4),(2,2),(4,1) \} \Leftrightarrow m \in \{ 3,4,6 \}$$
\end{proof}
Nous admetrons:
\begin{myth}
 Le quadrillage en hexagone est celui offrant le moins de chevauchement du point de vu WSN.
\end{myth}
%\begin{figure}[H]
%\centering
%\includegraphics[scale=0.7]{Etat_de_l'art/source/hexagone1.png}
%\caption{ Chevauchement d'un maillage hexagonale}
%\end{figure} 

Soit $P$ un plan sur lequel $n$ capteurs sont placés. Chaque capteur peut émettre des message avec un rayon compris entre 0 et $\gamma$ .Dans ce protocol, tout les noeuds auront un meme rayon d'émission fixe $R$.
Etant donnée une source $s$ et un message $M$ à broadcaster, nous devons placer les noeuds relais de facon a minimiser leur nombre $m$. Biensur leur nombre dépend directement du rayon $R$.
 

Dans un reseau hexagonale, nous avons les resultats suivants:
Soit $S$ l'aire du plan rectangulaire $P$. Connaissant $R$, nous pouvons calculer facilement le nombre de sommet pour couvrir le plan.
Soit $h$ le nombre d'hexagones couvrant $P$: 
Knowing r is the exact distance between two emitting
nodes, we can easily compute the necessary quantity of them
to cover the entire area. To do this, we just have to find how
many hexagons, denoted by h, fit on our area of surface S:
$$h \simeq \frac{Surface ( P)}{Surface(hexagone)}=\frac{2S}{3R^2 \sqrt{3}}$$

Comme il faut deux noeuds par hexagone,$$n=2h=\frac{4S}{3R^2 \sqrt{3}}= \frac{k}{R^2}$$ où $k=\frac{4S}{3 \sqrt{3}}$

Ainsi, le cout d'un broadcast vaut: $$C(1)(R)= n\cdot E(R)=\frac{k}{R^2}\cdot E(R)=\frac{k}{R^2}\cdot (R^\alpha+c)$$
Nous cherchons le rayon optimal minimisant $C(1)(R)$. Comme $\alpha\geq 2$, $c\geq 0$ et $R>0$, nous avons 4 cas possibles:
\begin{itemize}
 \item $\alpha=2$, $c=0$: nous avons $C(1)(R)=k$. $R_{opt}$ ne depend pas de $r$.
 \item $\alpha=2$, $c\neq0$: nous avons $C(1)(R)=k(1+cR^{-2})$. Au plus le rayon doit etre le plus gran possible: $R_{opt}=\gamma$
 \item $\alpha>2$, $c=0$: nous avons $C(1)(R)=kR^{\alpha-2}$. $R_{opt}=0$
 \item $\alpha>2$, $c\neq 0$: nous avons $C(1)(R)=k(R^{\alpha-2} + c R^{-2} ) $.\\
 En derivant, nous obtenons: $C'(1)(R)= k( (\alpha 2) R^{\alpha-3}- 2cR^{-3} ) $.\\
D'ou: $$R_{opt}=\sqrt[\alpha]{\frac{2c}{\alpha-2}}$$
\end{itemize}

Biensur, en pratique, il est impossible sur un topologie quelquonces d'extraire un tel maillage hexagonale. Cependant, l'idée des algorithmes de broadcast est de choisir les noeuds relais de façon à ce que leur ensemnble se rapproche 
au maximum d'un tel maillage.

\paragraph{TR-LBOP}
'Target Transmission Radius over LMST for Energy-Efficient Broadcast Protocol in Ad Hoc Networks':\cite{Ingelrest2004}.\\








\subsection{Synthèse}

\begin{figure}[H]
\centering
\includegraphics[scale=0.8]{Etat_de_l'art/source/classification}
\caption{ Graphe de synthèse}
\end{figure} 


