% !TEX encoding = UTF-8 Unicode
% !TEX root = ../rapport.tex

\chapter{Analyse et reflexion}\label{Analyse_et_reflexion}


\section{Critique de l'existant}



 


\section{Nos idées}

\subsection{Application du rayon optimal à DLBIP: TR-DLBIP}

DLBIP offre de bonne performences puisqu'il se base sur BIP qui est à ce jour l'algorithme le plus efficace en matiere de broadcast.
Nous l'avons modifié légerement en lui appliquant l'idée du rayon optimale.

\begin{algorithm}[H]
\caption{TR-DLBIP}
\label{algo_TRDLBIP}
\begin{algorithmic}
\STATE ENTREES : $G=(V,E)$ un graphe connexe, $s$ une source, un message $M$
\STATE SORTIE : message broadcasté
\STATE REQUIERE : connaissance du 2-voisinage
\STATE $R_{opt}=\sqrt[\alpha]{\frac{2c}{\alpha-2}}$
\STATE Mise à jour des couts des méssages: $ \forall i,j \in V, P'_{ij}=\frac{P_{ij}}{E_i}$
\STATE $s$ calcul $LBIP(s)=BIP-Tree(s,N_2(s))$
\STATE $s$ ajoute dans le paquet les identifiants des nœuds qui devront retransmettre le message
\STATE $s$ ajoute également les identifiants des nœuds qui devront être atteints par ceux qui retransmettent
\STATE $s$ broadcast <M,$s$> avec un rayon suffisant pour atteindre tous ses fils dans l'arbre
\IF{ $u$ reçoit <M,$v$> de $w$ }
	\IF{le paquet contient des instructions pour $u$}
		\STATE soit $v$ le nœud que $u$ doit atteindre d'après les instructions du paquet
		\STATE $u$ construit $LBIP(u)=BIP-Tree(s,N_2(u)\backslash \{$w,les sommets atteint par w,les sommets plus loin que $d_e(u,v)\})$ :
			\INDSTATE $u$ remplace dans le paquet les instructions pour les nœuds voisins
			\INDSTATE $u$ broadcast <M,$s$> avec pour rayon : 
			      \INDSTATE[2] - $R_{opt}$ si $\max\limits_{v\in N_u(1)\bigcap LBIP(u)}< R_{opt})$
			      \INDSTATE[2] - $\max\limits_{v\in N_u(1)\bigcap LBIP(u)}(d_e(u,v))$ sinon.
	\ENDIF
\ENDIF
\end{algorithmic}
\end{algorithm}


\subsection{}








