% !TEX encoding = UTF-8 Unicode
% !TEX root = ../rapport.tex

\chapter{Introduction}\label{intro}


Les réseaux sans fil font depuis plus d'une dizaine d'années partie intégrante de la vie quotidienne des entreprises, des particuliers, de l'industrie et d'autres organisations. Ils
représentent aujourd'hui une des briques de base sur lesquelles vont se fonder les systèmes
intelligents omniprésents qui vont constituer une des technologies de l'avenir. Cependant, la majeure partie de ces technologies sans fils, à commencer par le Wifi, elle est basée  sur des infrastructures fixes, limitant la mobilité des utilisateurs. Pour faciliter cette mobilité, il existe  un autre type de réseau, de plus en plus courant, qui  permet aux nœuds du réseau de communiquer directement entre eux sans nécessiter d’infrastructure : ce sont les réseaux ad hoc.

\begin{figure}[h]
\centering
\includegraphics[scale=0.8]{intro/reseauAdHoc.png}
\caption{\label{reseauAdHoc} Réseau ad hoc}
\end{figure}

On distingue donc deux principales classes de réseaux sans fils, les classiques structurés et les non structurés comme les réseaux ad hoc. Les réseaux ad hoc offrent la possibilité de connecter différents dispositifs sans avoir à préinstaller une infrastructure fixe comme dans les réseaux traditionnels. Dans les réseaux ad hoc, l’ensemble des nœuds communiquent directement entre eux (voir figure \ref{reseauAdHoc}). Nous allons nous intéresser à un type particulier de réseau ad hoc :  les réseaux de capteurs. Ces réseaux ont de nombreuses applications pratiques dans le médicale, la physique, la chimie, le multimédia, l'automobile, la climatologie…



\section{Capteurs}

Les capteurs sont des petites entités électroniques à faible coût qui ont pour but de récolter des informations dans leur environnement proche comme la température, la vitesse, le bruit, la pression, le mouvement, la chaleur ou la lumière… La valeur mesurée est convertie dans une représentation analogique ou numérique.

\begin{figure}[h]
\centering
\includegraphics[scale=0.8]{intro/imageCapteur}
\caption{\label{imageCapteur} Capteur sans fil}
\end{figure}

Il existe des \textbf{capteurs intelligents} (Smart Sensors) dans lesquels coexistent le(s) capteur(s) et les circuits de traitement et de communication. Leurs relations avec des couches de traitement supérieures vont bien au-delà d’une simple « transduction de signal ». Les capteurs intelligents sont des « capteurs d’informations » et non pas simplement des capteurs et des circuits de traitement du signal juxtaposés. De plus, les « Smart Sensors » ne sont pas des dispositifs banalisés car chacun de leurs constituants a été conçu dans l’objectif d’une application bien spécifique.

Lorsque nous parlerons de capteur plus loin dans ce rapport, il s'agira d'un capteur intelligent. Un tel capteur contient quatre unités de base (voir Figure \ref{archiCapteur}) : 
 
 \begin{figure}[h]
\centering
\includegraphics[scale=0.8]{intro/archiCapteur}
\caption{\label{archiCapteur} Architecture d’un capteur}
\end{figure}
 
 \begin{description}
\item[L'unité d’acquisition] est composée d’un capteur qui obtient des mesures sur les paramètres environnementaux et d’un convertisseur Analogique/Numérique qui convertit l’information relevée et la transmet à l’unité de traitement. La perception d'un capteur est limitée par un rayon de sensation (Rs). La  Figure \ref{percept} illustre ce principe.

\item[L'unité de traitement] est composée d’un processeur et d’une mémoire intégrant un système d’exploitation spécifique. Cette unité possède deux interfaces, une interface pour l’unité d’acquisition et une interface pour l’unité de communication. Elle acquiert les informations en provenance de l’unité d’acquisition et les envoie à l’unité de communication. Cette unité est chargée aussi d’exécuter les protocoles de communications qui permettent de faire collaborer le capteur avec d’autres capteurs. Elle peut aussi analyser les données captées.

\item[L'unité de communication] est l'unité responsable de toutes les émissions et réceptions de données via un support de communication radio. Elle peut être de type optique, ou de type radiofréquence. Fonctionnellement chaque capteur possède un rayon de communication (Rc). La  figure \ref{percept} montre la zone dans laquelle le capteur peut communiquer. Certains capteurs peuvent moduler leur rayon de communication.

\item[L'unité de contrôle d'énergie (batterie)] sert à alimenter tous les composants. Cependant, à cause de la taille réduite du capteur, la batterie est limitée et généralement irremplaçable. Ainsi, l’énergie est la ressource la plus précieuse puisqu’elle influe directement sur la durée de vie des capteurs.
\end{description}

Selon son domaine d'application, un capteur peut contenir des modules supplémentaires comme le système de positionnement GPS (Global Positioning System) ou un système lui permettant de se déplacer.

\begin{figure}[h]
\centering
\includegraphics[scale=0.8]{intro/percept}
\caption{\label{percept} Rayon de communication et de sensation}
\end{figure}




\section{Caractéristiques techniques des capteurs actuels}

\begin{description}
\item[Une faible puissance de calcul] : quand les ordinateurs peuvent avoir jusqu'à 4 processeurs, chacun cadencé à 3GHz, ou quand les derniers Smartphones peuvent fonctionner jusqu’à 800MHz, un capteur actuel est à peine plus puissant qu’une calculatrice graphique produite dans les années 90.

\item[Un espace de stockage mémoire limité] à quelques kilo-octets ou quelques méga-octets impose l'utilisation d'algorithmes distribués, localisés et collaboratifs.

\item[Une puissance radio limitée] : l’ordre de grandeur des portées actuellement atteignables par les principaux capteurs est d’une centaine de mètres en extérieur et de quelques dizaines de mètres en intérieur. Cette portée est largement dépendante de la fréquence utilisée et de l’environnement. Elle nécessite un routage multi-saut pour l’acheminement des données vers une entité de collecte : le puits. Les capteurs ne peuvent communiquer qu’avec leur voisinage direct qui va relayer les communications.

\item[Un débit faible] : les composants radio d’un capteur sont limités à quelques centaines de kilo-octets par seconde.

\item[Une réserve d’énergie réduite] : même s’il existe des mécanismes de recharge d’énergie, la durée de vie d’un capteur reste directement liée au niveau de sa batterie. Cette réserve d’énergie est partagée par chaque unité d’un capteur mais l’unité de communication va en consommer près de 95\% lors du fonctionnement actif du capteur. Les enjeux actuels portent donc sur :
	\begin{itemize}
	\item l’augmentation des capacités des batteries
	\item les dispositifs de transmission radio ultra-basse consommation
	\item les architectures basse consommation
	\item des mécanismes d’endormissement
	\item des protocoles de communication spécifiques
	\end{itemize}
\end{description}



\section{Réseau de capteurs sans fil (Wireless sensors network)}
Les réseaux de capteurs sans fil sont un type particulier de réseau ad-hoc. Ces réseaux sont formés d’une multitude de capteurs, capables de s’auto-organiser et ainsi de travailler pour la collecte, le partage et le traitement coopératif des informations sur leur environnement ; le tout sans intervention humaine. Ces dispositifs sont peu coûteux, mais peu performants. Depuis quelques décennies, le besoin croissant d’observer et de contrôler des phénomènes physiques tels que la température, la pression ou encore la luminosité a conduit au déploiement de nombreux réseaux de capteurs.

\begin{figure}[h]
\centering
\includegraphics[scale=0.8]{intro/WSN}
\caption{\label{WSN} Réseaux de capteurs}
\end{figure}

Dans l'exemple de la figure \ref{WSN}, les capteurs sont déployés d’une manière aléatoire dans une zone d’intérêt, et une station de base, située à l’extrémité de cette zone, est chargée de récupérer les données collectées par les capteurs. Lorsqu’un capteur détecte un événement pertinent, un message d’alerte est envoyé à la station de base par le biais d’une  communication entre les capteurs. Les données collectées sont traitées et analysées par des machines puissantes.



\subsection{Architecture d’un réseau de capteurs sans fil}



\begin{figure}[h]
\centering
\includegraphics[scale=0.8]{intro/archiWSN}
\caption{\label{archiWSN} architecture d’un réseau WSN}
\end{figure}

Les réseaux de capteurs sans fils sont construits autour des quatre principales entités suivantes :

\begin{description}
\item[Les capteurs] décrits précédemment.

\item[L’agrégateur] est en charge d’agréger les messages qu’il reçoit de plusieurs capteurs puis de les envoyer en un seul message au puits. Cette opération a pour principal but de limiter le trafic sur le réseau et donc de prolonger la durée de vie globale du réseau de capteur.

\item[Le puits] est le nœud final  du réseau. C’est à lui qu’est  envoyé l’ensemble des valeurs mesurées par le réseau.  Il peut arriver qu’il y ait plusieurs puits sur un même réseau de capteurs.

\item[La passerelle] est un dispositif qui a la particularité d’avoir deux interfaces réseau. Elle permet de relier le réseau de capteurs  sans fils  à un réseau plus traditionnel, typiquement l’internet. Habituellement  le réseau de capteurs  ne sert  qu’à  faire remonter les 
mesures, les applications traitant ces informations étant exécutées sur la machine 
de l’utilisateur final.
\end{description}



\subsection{Caractéristiques}
Les caractéristiques particulières des WSN modifient les critères de performances par rapport aux réseaux sans fil traditionnels. Dans les réseaux locaux filaires ou les réseaux cellulaires, les critères les plus pertinents sont le débit, la latence et la qualité de service car les nouvelles Activités telles que le transfert d’images,  le  transfert de vidéos, et la navigation sur Internet requièrent un débit important, une faible latence, et une bonne qualité de service. En revanche, dans les réseaux de capteurs conçus pour surveiller une zone d’intérêt,  la longévité du réseau est fondamentale. De ce fait, la conservation de l’énergie est devenue un critère de performance prépondérant et se pose en premier lieu tandis que les autres critères comme le débit ou l’utilisation de la bande passante sont devenus secondaires. Nous verrons dans ce qui suit quelque caractéristique des WSN.

\begin{itemize}

\item \textbf{Absence d’infrastructure} : d’absence d’infrastructure préexistante et de tout genre d’administration centralisée.

\item \textbf{Interférences} : les liens radio ne sont pas isolés, deux transmissions simultanées sur une même fréquence, ou utilisant des fréquences proches, peuvent interférer.

\item \textbf{Taille importante} : un réseau de capteurs peut contenir des milliers de nœuds.

\item \textbf{Hétérogénéité des nœuds} : plusieurs types différents connectant entre eux.

\item \textbf{Topologie dynamique} : les capteurs peuvent être attachés à des objets mobiles qui se déplacent d’une façon libre et arbitraire rendant ainsi la topologie du réseau fréquemment changeante.

\item \textbf{Contrainte d’énergie} : la caractéristique la plus critique dans les réseaux de capteurs est la modestie de ses ressources énergétiques (batterie).

\item \textbf{La capacité de stockage et la puissance de calcul} sont limitées dans un capteur.

\item \textbf{sécurité physique limitée} : Cela se justifie par les contraintes et limitations physiques.

\item \textbf{bande passante limitée} : la bande passante réservée à un nœud est limitée.

\item \textbf{Le faible coût du matériel} : Ce qui facilite une redondance des liens pour assurer une connexité du réseau en cas de panne d’un ou plusieurs capteurs

\end{itemize}

Une des forces d’un réseau de capteurs est sa capacité à être déployé rapidement dans des conditions difficiles. Le Positionnement des nœuds est donc aléatoire. Ces conditions de déploiement conduisent à des contraintes supplémentaires :

\begin{itemize}

\item \textbf{L’impossibilité de remplacer manuellement les nœuds} : pour position inconnu.

\item \textbf{Méconnaissance globale de la topologie du réseau} : due une forte densité et une forte cardinalité du réseau.

\end{itemize}


\subsection{Application des WSN}
Le fort potentiel d’application des réseaux de capteurs en fait un domaine de recherche très actif. Ces dernières années ont vu le passage d’applications anecdotiques à de véritables applications à large échelle. La réduction de plus en plus importante de la taille des capteurs, leur coût de plus en plus faible, ainsi que l’étendue du type de capteurs disponibles (thermique, optique, cinétique, chimique, etc.), permettent aux réseaux de capteurs d’envahir de très nombreux domaines d’application.

On dénombre 3 grandes familles d’applications pour réseaux de capteurs :
\begin{itemize}

\item \textbf{le relevé d’informations} : dans lequel chaque capteur collecte simplement les données et les envoie périodiquement au puits.

\item \textbf{la détection d’événements} : où chaque nœud décide s’il envoie des données au puits en fonction des données qu’il a collectées. Cette décision peut nécessiter au capteur un dialogue avec son voisinage.

\item \textbf{le contrôle intelligent} : dans lequel chaque capteur collecte différentes sortes de données, échange ses informations à travers le réseau et collabore pour former une description de l’état de l’objet ciblé.

\end{itemize}

Il a également été définit un ensemble de scénarios viables visé par les réseaux de capteurs

\begin{itemize}

\item Supervision de l'habitat écologique
\item Surveillance militaire et traque de cibles
\item Supervision des structures et des phénomènes liés à l’environnement (séisme, volcan)
\item Détection en réseau dans l'industrie et le commerce
\item Bâtiment intelligent
\item Domotique
\item Surveillance de l’environnement (volcan, météo…)
\item Santé  (La surveillance des fonctions vitales d'un organisme vivant)
\item Sécurité industrielle et Domestique

\end{itemize}


\subsection{Problématique et Défis}
Les perspectives d’application des réseaux de capteurs sont enthousiasmantes mais les défis qu’elles posent n’en sont pas moins nombreux et complexes. Parmi les problématiques cruciales, nous pouvons citer :
\begin{itemize}

\item \textbf{L’énergie} : cette contrainte impose de concevoir des protocoles économes en énergie.

\item \textbf{Le routage} : Le problème de routage consiste à déterminer un acheminement optimal des paquets à travers le réseau au sens d’un certain critère de performance (Energie par exemple).

\item \textbf{Sécurité} : La puissance de calcul limité des entités d’un capteur ouvre de véritables défis pour concevoir des algorithmes de cryptages distribués et des politiques de confiance spécifiques.

\item \textbf{La collecte de données} : récupérer les données des capteurs et les assemblés.

\item \textbf{Auto-configuration} : Une partie des applications visée appartient au domaine des applications domestiques. Il est donc important que le routage, l’intégration et l’adaptation à l’environnement soient transparents pour l’utilisateur.

\item \textbf{Autoréparation} : Nous venons de le voir, les capteurs sont parfois inaccessibles (intégrés dans un mur, installés chez un particulier, déployés dans une zone dangereuse, etc.) et parfois de conception peu sûre (faible coût de production). Une solution complète doit donc gérer efficacement la perte ou l’ajout d’un nœud dans son réseau.

\item \textbf{Localisation} : Il s’agit de concevoir des mécanismes de localisation réalistes vis-à-vis des contraintes et des applications propres aux réseaux de capteurs. Les solutions actuellement proposées reposent sur des solutions, soit imprécises, soit coûteuses en énergie ou en matériel.

\end{itemize}


