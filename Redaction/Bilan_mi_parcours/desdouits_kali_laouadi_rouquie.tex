\documentclass[a4paper]{article}

\usepackage[francais]{babel}
\usepackage[T1]{fontenc}
\usepackage[applemac]{inputenc}

\usepackage{geometry}
\usepackage{graphicx}
\usepackage[colorlinks=true]{hyperref}
\hypersetup{urlcolor=blue,linkcolor=black,citecolor=black,colorlinks=true}
\usepackage{listings}
\usepackage{amssymb}
\usepackage{setspace}
\usepackage{lscape}
\usepackage[final]{pdfpages}


\pagestyle{headings}
\thispagestyle{empty}
\geometry{a4paper,twoside,left=2.5cm,right=2.5cm,marginparwidth=1.2cm,marginparsep=3mm,top=2.5cm,bottom=2.5cm}

\begin{document}
\large
\setlength{\parskip}{6mm plus2mm minus2mm}
\setlength{\textwidth}{530pt}
\setlength{\parindent}{0cm}

21 janvier 2012 \hfill M1 Informatique


{\centering \Large \bfseries Analyse et conception d'algorithmes �conomes en �nergie dans les r�seaux de capteurs \par}

{\emph{Nom du groupe} : WSN\\ \\
\emph{�tudiants} :
\begin{description}
\item Chlo� DESDOUITS \hfill chloe.desdouits@etud.univ-montp2.fr
\item Sofiane Zahir KALI \hfill zahir.kali@etud.univ-montp2.fr
\item Rabah LAOUADI \hfill rabah.laouadi@etud.univ-montp2.fr
\item Samuel ROUQUIE \hfill samuel.rouquie@etud.univ-montp2.fr
\end{description}
~\\
\emph{Encadrante} : Anne-Elisabeth BAERT
\par}

{\bfseries Les t�ches r�alis�es � ce jour sont les suivantes :}\\
\begin{itemize}\addtolength{\itemsep}{0.2cm}
\renewcommand\labelitemi{\textbullet}
	\item Lecture et �criture des r�sum�s de dix articles
	\begin{description}
	\renewcommand{\makelabel}[1]{\normalsize \textbf{#1}}
		\item[Chlo� Desdouits] \cite{Champ2009DLBIP} \cite{Champ2009lifetime} \cite{Chang2000} \cite{Shah2002}
		\item[Sofiane Zahir Kali] \cite{Champ2009DLBIP} \cite{Champ2009lifetime} \cite{Cartigny2003LMEB} \cite{Cartigny2005}
		\item[Rabah Laouadi] \cite{Champ2009DLBIP} \cite{Champ2009lifetime} \cite{Ingelrest2005}
		\item[Samuel Rouquie] \cite{Champ2009DLBIP} \cite{Champ2009lifetime} \cite{Abba2006} \cite{Dietrich2009} \cite{Agarwal2005}
	\end{description}
	\item �criture de l'introduction du rapport (Rabah Laouadi et Chlo� Desdouits).
	\item Synth�se des connaissances acquises ; �criture de l'�tat de l'art du rapport (Samuel Rouquie).
	\item Programmation d'algorithmes sous WSNET (simulateur �v�nementiel de r�seaux)
	\begin{description}
	\renewcommand{\makelabel}[1]{\normalsize \textbf{#1}}
		\item[Chlo� Desdouits] $FA$ \cite{Chang2000}
		\item[Sofiane Zahir Kali] $RNG$ \cite{Cartigny2005}
		\item[Rabah Laouadi] $LBIP$ \cite{Ingelrest2008}, $MPR$ \cite{Lehsaini2007} et $NES$ \cite{Stojmenovic2002}
	\end{description}
\end{itemize}
~\\
{\bfseries Voici les probl�mes que nous avons rencontr�s :}\\
\begin{itemize}\addtolength{\itemsep}{0.3cm}
\renewcommand\labelitemi{\textbullet}
	\item La mise en place et l'utilisation d'un outil de travail collaboratif : git.
	\item L'utilisation de \LaTeX{} pour la r�daction des documents.
	\item La prise en main de WSNET car ce framework est en langage C mais �v�nementiel et poss�de une architecture particuli�re.
	\item La probl�matique des algorithmes �conomes en �nergie dans les r�seaux de capteurs est une probl�matique large. Nous devons donc d�cider de traiter une sous-partie de cette probl�matique (par exemple les algorithmes de broadcast).
\end{itemize}
~\\
{\bfseries Les t�ches qui nous reste � effectuer sont les suivantes :}\\
\begin{itemize}\addtolength{\itemsep}{0.2cm}
\renewcommand\labelitemi{\textbullet}
	\item Lecture d'autres articles.
	\item R�daction de la synth�se des articles que nous aurons lu.
	\item Programmation de quelques autres algorithmes existants ($DLBIP$, $TRLOB$�).
	\item Prototypage de notre propre algorithme.
	\item R�daction de la suite du rapport.
\end{itemize}

\bibliographystyle{plain}
\bibliography{../Bib}

\end{document}
